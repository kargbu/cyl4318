\documentclass[fc]{tarea}

\usepackage[style=mexican]{csquotes}

\newcommand{\al}{\symscr{A}}
\newcommand{\lan}{\symscr{L}}
\newcommand{\sis}{\symup{SF}}
\newcommand{\expresiones}{\symup{Exp}}
\NewDocumentCommand{\expr}{O{\al}}{\expresiones(#1)}

%conjuntos
\providecommand\st{}
\newcommand\SetSymbol[1][]{%
\nonscript\:#1\vert
\allowbreak
\nonscript\:
\mathopen{}}
\DeclarePairedDelimiterX\set[1]{\{}{\}}{%
\renewcommand\st{\SetSymbol[\delimsize]}
#1
}

\examen{Tarea 2}
\prof{Karina G. Buendía}
\materia{Conjuntos y lógica}
\alum{}

\begin{document}
\HojaExamen{}{e}
\chapter{Ejercicios de lógica}

\begin{exercise}
  Demuestra que si \(\Gamma\vdash\alpha\) entonces existe un subconjunto finito
  \(\Delta\subseteq\Gamma\) tal que \(\Delta\vdash\alpha\).
\end{exercise}
\begin{solution}
  Si \(\Gamma\vdash\alpha\) entonces existe una sucesión finita de fórmulas \(\alpha_1,\ldots,\alpha_n\) que satisfacen las condiciones de la definición~\ref{def:ded}. Consideramos
  \(\Delta=\Gamma\cap\{\alpha_1,\ldots,\alpha_n\}\). Notamos que \(\Delta\) es un conjunto finito y que \(\Delta\vdash\alpha\). Lo último es porque la misma lista finita \(\alpha_1,\ldots,\alpha_n\) es una deducción de \(\alpha\) a partir de \(\Delta\).
\end{solution}

\begin{exercise}
  Demuestra que la cerradura deductiva es un operador de cerradura, es decir,
  \begin{tasks}
    \task \(\Gamma\subseteq\overline{\Gamma}\),
    \task si \(\Delta\subseteq\Gamma\) entonces
      \(\overline{\Delta}\subseteq\overline{\Gamma}\),
    \task \(\overline{\overline{\Gamma}}=\overline{\Gamma}\)
  \end{tasks}
\end{exercise}

\begin{exercise}
  Considera el alfabeto \(\al=\set{a,b,\circ}\), la definición de fórmula dada por \(\Phi=\set{\alpha\in\expr\st\alpha\text{ empieza con }a}\) y la regla de inferencia
  \[
    R:
    \begin{array}{c}
      A,B\\
      \midrule
      A\circ B
    \end{array}.
  \]
  Con esto definimos el lenguaje \(\lan=(\al,\Phi)\) y al sistema formal \(\sis=(\lan,\set{R})\). Sea 
  \(T=\set{\alpha\st\alpha\text{ no tiene ocurrencias de }\circ}\).
  \begin{tasks}
    \task Demuestra que \(T\) es correcta y no completa respecto a la propiedad \enquote{no tener \(b\) después de cada \(\circ\)}.
    \task Muestra que \(T\) es completa y no correcta respecto a la propiedad \enquote{tener \(aa\) después de cada \(\circ\)}.
  \end{tasks}
\end{exercise}

\begin{exercise}
  Considera el sistema formal dado por lo siguiente: el alfabeto es
  \(\al=\set{a,b,c}\), el conjunto de fórmulas es 
  \(\Phi=\set{\alpha\in\expr\st\alpha\text{ empieza con } a}\) y las reglas de inferencia son
  \[
    R_1:
    \begin{array}{c}
      axb\\
      \midrule
      axbc
    \end{array}
    \quad
    R_2:
    \begin{array}{c}
      ax\\
      \midrule
      axx
    \end{array}
    \quad
    R_3:
    \begin{array}{c}
      axbbby\\
      \midrule
      axcy
    \end{array}
    \quad
    R_4:
    \begin{array}{c}
      axccy\\
      \midrule
      axy
    \end{array},
  \]
  donde \(x\) y \(y\) representan sucesiones finitas de símbolos de \(\al\). Considera \(\Gamma=\set{ab}\) y demuestra lo siguiente:
  \begin{tasks}(2)
    \task \(\Gamma\vdash abcc\),
    \task \(\Gamma\vdash acbbc\),
    \task ¿es posible obtener \(\Gamma\vdash ac\)?
  \end{tasks}
\end{exercise}

\begin{exercise}
  Usa el hecho de que \(2 = \{{0, 1}\}\) para demostrar que si \(\alpha\) y \(\beta\) son proposiciones tales que \(e(\alpha)=1\) si y solo si \(e(\beta)=1\) para toda evaluación \(e: From \to 2\), entonces \(\alpha\equiv \beta\).
\end{exercise}

\begin{exercise}
Sea \(\mathbb{P}\) un conjunto de letras proposicionales. Consideramos el conjunto de todas las posibles evaluacione \(e: From \to 2\)una función. Demuestra que toda proposición \(e\) induce una partición en dos pedazos del conjunto \(2^]{\mathbb{P}}\) .
\end{exercise}

\begin{exercise} Demuestra que \(\{\neg,\;\land\}\) es un conjunto mínimo de conectivos, es decir, que el resto de conectivos se pueden definir en términos de ellos dos. También muestra que \(\{\neg,\;\iff\}\) y \(\{\lor,\;\land\}\) no son conjuntos mínimos de conectivos, es decir, hay al menos un conectivo que no se puede definir usando sólo los conectivos de cada conjunto.
\end{exercise}

\begin{exercise}
  Demuestra que para cada fórumula \(\alpha\) y \(\beta\) se sigue:
  \begin{enumerate}
   \item \(\vdash \neg \alpha \rightarrow (\alpha \rightarrow \beta)\)
   \item \(\vdash (\alpha \rightarrow \beta) \rightarrow (\neg \beta \rightarrow \neg \alpha)\)
  \end{enumerate}
\end{exercise}

\begin{exercise}
  Usa el teorema de las \enquote{primas} para demostrar que de la deducción \(\set{\alpha,\;\neg \beta}\vdash (\neg \alpha \rightarrow \beta)\), con \(\gamma \equiv (\neg \alpha \rightarrow \beta)\), se puede demostrar que \(\set{\alpha',\;\beta'}\vdash \gamma'\).
\end{exercise}

\begin{exercise}
  Decimos que una teoría \(T\) es consistente si existe una fórmula \(\alpha\) tal que \(T\not \vdash \alpha\). Demuestra que el calculo de proposiciones es consistente.
\end{exercise}

\end{document}