\documentclass[fc]{tarea}

\usepackage[style=mexican]{csquotes}

\newcommand{\al}{\symscr{A}}
\newcommand{\lan}{\symscr{L}}
\newcommand{\sis}{\symup{SF}}
\newcommand{\expresiones}{\symup{Exp}}
\NewDocumentCommand{\expr}{O{\al}}{\expresiones(#1)}

%conjuntos
\providecommand\st{}
\newcommand\SetSymbol[1][]{%
\nonscript\:#1\vert
\allowbreak
\nonscript\:
\mathopen{}}
\DeclarePairedDelimiterX\set[1]{\{}{\}}{%
\renewcommand\st{\SetSymbol[\delimsize]}
#1
}

\examen{Tarea 3}
\prof{Karina García Buendía y José Dosal}
\materia{Conjuntos y lógica}
\alum{}

%\xsimsetup{solution/print = true}

\begin{document}
\HojaExamen{}{e}
\chapter{Ejercicios de lógica}

\begin{exercise}
  Demuestra que si \(\Gamma\vdash\alpha\) entonces existe un subconjunto finito
  \(\Delta\subseteq\Gamma\) tal que \(\Delta\vdash\alpha\).
\end{exercise}
\begin{solution}
  Si \(\Gamma\vdash\alpha\) entonces existe una sucesión finita de fórmulas \(\alpha_1,\ldots,\alpha_n\) que satisfacen las condiciones de la definición~\ref{def:ded}. Consideramos
  \(\Delta=\Gamma\cap\{\alpha_1,\ldots,\alpha_n\}\). Notamos que \(\Delta\) es un conjunto finito y que \(\Delta\vdash\alpha\). Lo último es porque la misma lista finita \(\alpha_1,\ldots,\alpha_n\) es una deducción de \(\alpha\) a partir de \(\Delta\).
\end{solution}

\begin{exercise}
  Demuestra que la cerradura deductiva es un operador de cerradura, es decir,
  \begin{tasks}
    \task \(\Gamma\subseteq\overline{\Gamma}\),
    \task si \(\Delta\subseteq\Gamma\) entonces
      \(\overline{\Delta}\subseteq\overline{\Gamma}\),
    \task \(\overline{\overline{\Gamma}}=\overline{\Gamma}\)
  \end{tasks}
\end{exercise}

\begin{exercise}
  Considera el alfabeto \(\al=\set{a,b,\circ}\), la definición de fórmula dada por \(\Phi=\set{\alpha\in\expr\st\alpha\text{ empieza con }a}\) y la regla de inferencia
  \[
    R:
    \begin{array}{c}
      A,B\\
      \midrule
      A\circ B
    \end{array}.
  \]
  Con esto definimos el lenguaje \(\lan=(\al,\Phi)\) y al sistema formal \(\sis=(\lan,\set{R})\). Sea 
  \(T=\set{\alpha\st\alpha\text{ no tiene ocurrencias de }\circ}\).
  \begin{tasks}
    \task Demuestra que \(T\) es correcta y no completa respecto a \enquote{no tener \(b\) después de cada \(\circ\)}.
    \task Muestra que \(T\) es completa y no correcta respecto a \enquote{tener \(aa\) después de cada \(\circ\)}.
  \end{tasks}
\end{exercise}

\begin{exercise}
  Considera el sistema formal dado por lo siguiente: el alfabeto es
  \(\al=\set{a,b,c}\), el conjunto de fórmulas es 
  \(\Phi=\set{\alpha\in\expr\st\alpha\text{ empieza con } a}\) con las reglas de inferencia siguientes
  \[
    R_1:
    \begin{array}{c}
      axb\\
      \midrule
      axbc
    \end{array}
    \quad
    R_2:
    \begin{array}{c}
      ax\\
      \midrule
      axx
    \end{array}
    \quad
    R_3:
    \begin{array}{c}
      axbbby\\
      \midrule
      axcy
    \end{array}
    \quad
    R_4:
    \begin{array}{c}
      axccy\\
      \midrule
      axy
    \end{array},
  \]
  donde \(x\) y \(y\) representan sucesiones finitas de símbolos de \(\al\). Considera \(\Gamma=\set{ab}\) y demuestra lo siguiente:
  \begin{tasks}(2)
    \task \(\Gamma\vdash abcc\),
    \task \(\Gamma\vdash acbbc\),
    \task ¿es posible obtener \(\Gamma\vdash ac\)?
  \end{tasks}
\end{exercise}

\iffalse
\begin{exercise}
  Usa el hecho de que \(2 = \{{0, 1}\}\) para demostrar que si \(\alpha\) y \(\beta\) son proposiciones tales que \(e(\alpha)=1\) si y solo si \(e(\beta)=1\) para toda evaluación \(e: From \to 2\), entonces \(\alpha\equiv \beta\).
\end{exercise}

\begin{exercise}
Sea \(\mathbb{P}\) un conjunto de letras proposicionales. Consideramos el conjunto de todas las posibles evaluacione \(e: From \to 2\) una función. Demuestra que toda proposición \(e\) induce una partición en dos pedazos del conjunto \(2^]{\mathbb{P}}\) .
\end{exercise}
\fi

\begin{exercise} Demuestra que \(\{\neg,\;\land\}\) es un conjunto mínimo de conectivos, es decir, que el resto de conectivos se pueden definir en términos de ellos dos. También muestra que \(\{\neg,\;\iff\}\) y \(\{\lor,\;\land\}\) no son conjuntos mínimos de conectivos, es decir, hay al menos un conectivo que no se puede definir usando sólo los conectivos de cada conjunto.
\end{exercise}

\iffalse
\begin{exercise}
  Demuestra que para cada fórmula \(\alpha\) y \(\beta\) se sigue:
  \begin{enumerate}
   \item \(\vdash \neg \alpha \rightarrow (\alpha \rightarrow \beta)\)
   \item \(\vdash (\alpha \rightarrow \beta) \rightarrow (\neg \beta \rightarrow \neg \alpha)\)
  \end{enumerate}
\end{exercise}
\fi

\begin{exercise}
  Usa el teorema de las \enquote{primas} para demostrar que de la deducción \(\set{\alpha,\;\neg \beta}\vdash (\neg \alpha \rightarrow \beta)\), con \(\gamma \equiv (\neg \alpha \rightarrow \beta)\), se puede demostrar que \(\set{\alpha',\;\beta'}\vdash \gamma'\).
\end{exercise}

\begin{exercise}
  Decimos que una teoría \(T\) es consistente si existe una fórmula \(\alpha\) tal que \(T\not \vdash \alpha\). Demuestra que el cálculo de proposiciones es consistente.
\end{exercise}


\begin{exercise}
    Realiza las siguientes deducciones en el cálculo de proposiciones.
    \begin{tasks}(2)
      \task $\vdash \neg \neg \beta \rightarrow \beta$ 
      \task $\vdash \beta \rightarrow \neg \neg \beta$
      \task $\vdash \beta \rightarrow (\neg \gamma \rightarrow \neg (\beta \rightarrow \gamma))$
      \task $\vdash (\alpha \rightarrow \beta) \rightarrow (\neg \beta \rightarrow \neg \alpha)$ 
      \task $\{\alpha \rightarrow \beta, \beta \rightarrow \gamma\} \vdash \alpha \rightarrow \gamma$ 
    \end{tasks}
  \end{exercise}
    
    \begin{solution}
       \text{a)}
    \begin{tabular}{l|l|l}
    \textbf{No.} & \textbf{Fórmula} & \textbf{Justificación} \\
    \hline
    1 & $\neg \neg \beta$ & \text{Premisa por el TD} \\
    2 & $\neg \beta \rightarrow \neg \beta$ & \text{Teorema ($\vdash P \rightarrow P$)} \\
    3 & $(\neg \beta \rightarrow \neg \beta) \rightarrow ((\neg \beta \rightarrow \beta) \rightarrow \beta)$ & \text{Axioma 3} \\
    4 & $(\neg \beta \rightarrow \beta) \rightarrow \beta$ & \text{MP (2, 3)} \\
    5 & $\neg \neg \beta \rightarrow (\neg \beta \rightarrow \neg \neg \beta)$ & \text{Axioma 1} \\
    6 & $\neg \beta \rightarrow \neg \neg \beta$ & \text{MP (1, 5)} \\
    7 & $\beta$ & \text{A partir de (4) y (6), implica $\neg \beta \rightarrow \beta$ (ej. $\gamma$). (Axioma A3)} \\
    \hline
    8 & $\neg \neg \beta \rightarrow \beta$ & \text{TD (1-7)}
    \end{tabular}
    \end{solution}

    \iffalse
    \item[\textbf{c)}] \textbf{Negación de la Implicación: $\vdash \beta \rightarrow (\neg \gamma \rightarrow \neg (\beta \rightarrow \gamma))$}
    
    Esta fórmula es un teorema: $\vdash \beta \rightarrow (\neg \gamma \rightarrow \neg (\beta \rightarrow \gamma))$ es equivalente a $\beta, \neg \gamma \vdash \neg (\beta \rightarrow \gamma)$. Esto se demuestra por Reducción al Absurdo, donde $\beta, \neg \gamma, (\beta \rightarrow \gamma) \vdash \gamma \land \neg \gamma$, lo cual es una contradicción.

    \item[\textbf{d)}] \textbf{Ley de Contraposición: $\vdash (\alpha \rightarrow \beta) \rightarrow (\neg \beta \rightarrow \neg \alpha)$}
    
    \begin{proof}
    Demostraremos $\alpha \rightarrow \beta, \neg \beta \vdash \neg \alpha$.
    \begin{tabular}{l|l|l}
    \textbf{No.} & \textbf{Fórmula} & \textbf{Justificación} \\
    \hline
    1 & $\alpha \rightarrow \beta$ & \text{Premisa} \\
    2 & $\neg \beta$ & \text{Premisa} \\
    3 & $\neg \neg \alpha$ & \text{Premisa (para Reducción al Absurdo/TD)} \\
    4 & $\neg \neg \alpha \rightarrow \alpha$ & \text{Eliminación de Doble Negación ($\neg \neg \alpha \rightarrow \alpha$)} \\
    5 & $\alpha$ & \text{MP (4, 3)} \\
    6 & $\beta$ & \text{MP (1, 5)} \\
    7 & $\beta \rightarrow \neg \neg \beta$ & \text{Introducción de Doble Negación ($\beta \rightarrow \neg \neg \beta$)} \\
    8 & $\neg \neg \beta$ & \text{MP (6, 7)} \\
    9 & $\neg \beta \rightarrow (\neg \neg \beta \rightarrow \neg \beta)$ & \text{Instancia de (A1)} \\
    10 & $\neg \neg \beta \rightarrow \neg \beta$ & \text{MP (2, 9)} \\
    11 & $\neg \beta$ & \text{MP (8, 10)} (\text{Contradicción: $\neg \beta$ y $\beta$}) \\
    \hline
    12 & $\neg \alpha$ & \text{TD (3-11), dado que se obtiene una contradicción} \\
    \hline
    13 & $\neg \beta \rightarrow \neg \alpha$ & \text{TD (2-12)} \\
    \hline
    14 & $(\alpha \rightarrow \beta) \rightarrow (\neg \beta \rightarrow \neg \alpha)$ & \text{TD (1-13)}
    \end{tabular}
    \end{proof}

    \item[\textbf{e)}] \textbf{Silogismo Hipotético: $\{\alpha \rightarrow \beta, \beta \rightarrow \gamma\} \vdash \alpha \rightarrow \gamma$}
    
    \begin{proof}
    Demostraremos $\alpha \rightarrow \beta, \beta \rightarrow \gamma, \alpha \vdash \gamma$.
    \begin{tabular}{l|l|l}
    \textbf{No.} & \textbf{Fórmula} & \textbf{Justificación} \\
    \hline
    1 & $\alpha \rightarrow \beta$ & \text{Premisa} \\
    2 & $\beta \rightarrow \gamma$ & \text{Premisa} \\
    3 & $\alpha$ & \text{Premisa (para TD} \\
    4 & $\beta$ & \text{MP (1, 3)} \\
    5 & $\gamma$ & \text{MP (2, 4)} \\
    \hline
    6 & $\alpha \rightarrow \gamma$ & \text{TD (3-5)}
    \end{tabular}
    \end{proof}
    \fi
\chapter{Ejercicios de conjuntos}

    \begin{exercise}
      Demuestra las siguientes propiedades de los números naturales.
      \begin{tasks}
        \task Para todo $m,n,p \in \mathbb{N}$,
          $$m<n\; \Leftrightarrow\; m + p < n + p $$.
        \task Para todo $m,n \in \mathbb{N}$,
          $$m + n = 0\; \Leftrightarrow\; m = 0\; \text{y}\; n = 0$$.
      \end{tasks}
    \end{exercise}

    \begin{exercise}    
      Sea $g$ una función en $A \times \mathbb{N}$ y sea $a \in A$. Entonces hay una única función $f$ tal que:
      \begin{tasks}
        \task $f(0) = a,$
        \task $f(S(n)) = g(f(n),n)$ para todo $n$ tal que $S(n) \in \mathrm{dom}\,f$,
        \task o bien $\mathrm{dom}\,f = \mathbb{N}$, o $\mathrm{dom}\,f = S(k)$ donde $k$ es el elemento máximo de 
        $\{\,k \in \mathbb{N} : g(f(k),k) \notin A \,\}.$
      \end{tasks}
    \end{exercise}

\end{document}