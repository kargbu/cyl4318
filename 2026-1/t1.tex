\documentclass[fc]{tarea}

\usepackage[style=mexican]{csquotes}

\newcommand{\menos}{\backslash}
\newcommand{\al}{\symscr{A}}
\newcommand{\lan}{\symscr{L}}
\newcommand{\sis}{\symup{SF}}
\newcommand{\expresiones}{\symup{Exp}}
\NewDocumentCommand{\expr}{O{\al}}{\expresiones(#1)}

%conjuntos
\providecommand\st{}
\newcommand\SetSymbol[1][]{%
\nonscript\:#1\vert
\allowbreak
\nonscript\:
\mathopen{}}
\DeclarePairedDelimiterX\set[1]{\{}{\}}{%
\renewcommand\st{\SetSymbol[\delimsize]}#1}
\DeclarePairedDelimiterX\bset[1]{\lbrack}{\rbrack}{#1}

\examen{Tarea 1}
\prof{Karina G. Buendía y José Dosal}
\materia{Conjuntos y lógica}
\alum{}

%\xsimsetup{solution/print = true}


\begin{document}
\HojaExamen{}{e}
\begin{exercise}
    Demuestra que el conjunto de todos los $x$ tales que $x \in A$ y $x \notin B$ existe y que es único.

\end{exercise}

\begin{solution}
Por el axioma de comprensión tenemos que $P(x):= x \notin B$. Para cualquier conjunto $A$ hay un conjunto $B$,
tal que si $x\in B \Leftrightarrow$ $x \in A$ y $P(x)$. Construimos el conjunto $C = \{x \in A| x\notin B\}$.
Supongamos que existe un $C'=\{x \in A| x\notin B\}$. Por el axioma de extensión, debemos mostrar que 
$C \subseteq C'$ y $C' \subseteq C$. Sea $x \in C$ si y solo si $x \in A$ y $x\notin B$
pero esta propiedad es la que cumplen los elementos de $C'$. Por lo tanto $C = C'$.
\end{solution}

\begin{exercise}
    Demuestre que para cualquier conjunto $X$ hay algún $a\notin X$.
\end{exercise}

\begin{solution}
La prueba es por contradicción. Supongamos que existe, para todo $a$ elemento $a \in X$ (lo contiene todo).
Por el axioma de comprensión, tenemos un subconjunto de $X$, digamos $A = \{a | P(x)\}$ donde $P(x):= x\notin x$
Este conjunto es igual al de la paradoja de Russel. Cuando nos preguntamos si $A \in A$ nos lleva a una contradicción.
Por lo que suponer que existe tal conjunto no puede suceder.
\end{solution}

\begin{exercise}
    Demuestre que $A\subseteq \{A\}$ si y solo si $ A = \varnothing$.
\end{exercise}

\begin{solution}
   Para la implicación $A = \varnothing$, entonces $A\subseteq \{A\}$. Es trivial. Para la dirección contraria
   tenemos que demostrarlo por contradicción. Supongamos que $A\subseteq \{A\}$ y $A \neq \varnothing$.
\end{solution}

\begin{exercise}
    Demuestre que si $A \subseteq B$, entonces $P(A) \subseteq P(B)$
\end{exercise}

\begin{exercise}
  Demuestre que $A \subseteq C$ si y solo si $A \cup (B \cap C) = (A \cup B) \cap C$  
\end{exercise}

\begin{exercise}
    Si $E$ es un conjunto que contiene a $A \cup B$, entonces:
    \begin{tasks}
    \task $E \menos (E \menos A) = A$
    \task $E \menos \varnothing = E, E \menos E= \varnothing$.
    \end{tasks}
\end{exercise}

\begin{exercise}
    Para todo conjunto $A$, $B$ y $C$ se cumple lo siguiente:
    \begin{tasks}
    \task $A \triangle \varnothing = A$
    \task $A \triangle A = \varnothing$
    \task Si $A \triangle B = A \triangle C$, entonces $B = C$
    \end{tasks}
\end{exercise}

\begin{exercise}
    Sea $F$ una familia de conjuntos. Pruebe que $\bigcup F = \varnothing$ si y solo si $F = \varnothing$
    o $A \in F$ implica $A = \varnothing$.
\end{exercise}

\begin{exercise}
    Demuestre que la unión y la intersección generalizada satisface la siguiente forma de asociación:
    
     \begin{tasks}
     \task $\bigcup \set{ A_\alpha | \alpha \in \bigcup I} = \bigcup_{I \in I} (\bigcup_{\alpha \in I}A_\alpha)$
     \task $\bigcap \set{ A_\alpha | \alpha \in \bigcap I} = \bigcap_{I \in I} (\bigcap_{\alpha \in I}A_\alpha)$
     \end{tasks}
\end{exercise}

\begin{exercise}
    Demuestra lo siguiente:
    \begin{tasks}
        \task $\bigcup_\alpha$ distribuye sobre $\cap$ y $\bigcup_\alpha$ distribuye sobre $\cup$,
        $$\bset{\bigcap_{\alpha \in I} A_\alpha} \cup \bset{\bigcap_{\beta \in J} B_\beta} = \bigcap \set{A_\alpha \cup B_\beta | (\alpha,\beta) \in I\times J}$$
        \task Si el complemento es tomado respecto a $X$, entonces
        $$X\menos \bigcap \set{A_\alpha | \alpha \in I} = \bigcup \set{X\menos A_\alpha | \alpha \in I}$$
        \task $\bigcup_\alpha$ y $\bigcap_\alpha$ distribuyen sobre el producto cartesiano
        $$\bset{\bigcap_{\alpha \in I} A_\alpha} \times \bset{\bigcap_{\beta \in J} B_\beta} = \bigcap \set{A_\alpha \times B_\beta | (\alpha, \beta)\in I \times J}$$
    \end{tasks}
\end{exercise}

\end{document}