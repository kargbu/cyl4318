\documentclass[fc]{tarea}

\usepackage[style=mexican]{csquotes}

\newcommand{\menos}{\backslash}
\newcommand{\al}{\symscr{A}}
\newcommand{\lan}{\symscr{L}}
\newcommand{\sis}{\symup{SF}}
\newcommand{\expresiones}{\symup{Exp}}
\NewDocumentCommand{\expr}{O{\al}}{\expresiones(#1)}

%conjuntos
\providecommand\st{}
\newcommand\SetSymbol[1][]{%
\nonscript\:#1\vert
\allowbreak
\nonscript\:
\mathopen{}}
\DeclarePairedDelimiterX\set[1]{\{}{\}}{%
\renewcommand\st{\SetSymbol[\delimsize]}#1}
\DeclarePairedDelimiterX\bset[1]{\lbrack}{\rbrack}{#1}

\examen{Ejemplos de productos arbitrarios}
\prof{Karina G. Buendía y José Dosal}
\materia{Conjuntos y lógica}
\alum{}

%\xsimsetup{solution/print = true}

\begin{document}

\begin{tasks}
    \task Calcular el producto cartesiano $\prod_{i=1}^{3} A_i$ donde $A_1=\set{a, b}$,\;
    $A_2=\set{x,y,z}$ y $A_3=\set{Pepe}$ es fácil. $\prod_{i=1}^{3} A_i = \set{(a,x,Pepe),\;
(a,y,Pepe),\;
(a,z,Pepe),\;
(b,x,Pepe),\;
(b,y,Pepe),\;
(b,z,Pepe)}$. Pero cada uno de estos elementos del producto es una función, dicho de otro modo,
el producto calcula todas las funciones cuyo dominio es el conjunto $I$ de índices y codominio
la unión de los $A_i$'s, por ejemplo,
la tupla $(a, x, Pepe)$ representa a la función $f_1$ tal que $f_ 1 (1)=a$, $f_1 (2)=x$ y 
$f_1 (3)=Pepe$.
El producto cartesiano es el conjunto que contiene exactamente estas seis funciones, que 
podemos describir formalmente como conjuntos de pares ordenados:\\
$f_1=\set{(1,a),(2,x),(3,Pepe)}\\
f_2=\set{(1,a),(2,y),(3,Pepe)}\\
f_3=\set{(1,a),(2,z),(3,Pepe)}\\
f_4=\set{(1,b),(2,x),(3,Pepe)}\\
f_5=\set{(1,b),(2,y),(3,Pepe)}\\
f_6=\set{(1,b),(2,z),(3,Pepe)}$

\end{tasks}

\end{document}