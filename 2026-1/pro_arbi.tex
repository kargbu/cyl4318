\documentclass[fc]{tarea}

\usepackage[style=mexican]{csquotes}

\newcommand{\menos}{\backslash}
\newcommand{\al}{\symscr{A}}
\newcommand{\lan}{\symscr{L}}
\newcommand{\sis}{\symup{SF}}
\newcommand{\expresiones}{\symup{Exp}}
\NewDocumentCommand{\expr}{O{\al}}{\expresiones(#1)}

\providecommand\st{}
\newcommand\SetSymbol[1][]{%
\nonscript\:#1\vert
\allowbreak
\nonscript\:
\mathopen{}}
\DeclarePairedDelimiterX\set[1]{\{}{\}}{%
\renewcommand\st{\SetSymbol[\delimsize]}#1}
\DeclarePairedDelimiterX\bset[1]{\lbrack}{\rbrack}{#1}


\examen{Ejemplos de productos arbitrarios}
\prof{Karina G. Buendía y José Dosal}
\materia{Conjuntos y lógica}
\alum{}

\begin{document}
\HojaExamen{}{e}
\begin{tasks}

\task El producto cartesiano de la familia de conjuntos $\al = \set{A_i}_{i \in \set{1,2,3}}$
    se denota como $\prod_{i=1}^{3} A_i$ donde $A_1=\set{a, b}$,\;
    $A_2=\set{x,y,z}$ y $A_3=\set{Pepe}$.
    
    Es fácil ver que el producto es el conjunto:
    \[ \prod_{i=1}^{3} A_i = 
    \set{ \begin{aligned}
        &(a,x,\text{Pepe}),\; (a,y,\text{Pepe}),\; (a,z,\text{Pepe}), \\
        &(b,x,\text{Pepe}),\; (b,y,\text{Pepe}),\; (b,z,\text{Pepe})
    \end{aligned} }
    \]
    
    Pero cada uno de estos elementos es una función. Dicho de otro modo, el producto calcula todas las funciones cuyo dominio es el conjunto de índices $I$ y cuyo codominio es la unión de los $A_i$. Por ejemplo, la tupla $(a, x, \text{Pepe})$ representa a la función $f_1$ tal que $f_1(1)=a$, $f_1(2)=x$ y $f_1(3)=\text{Pepe}$.
    
    El producto cartesiano es el conjunto que contiene exactamente estas seis funciones, que podemos describir formalmente como conjuntos de pares ordenados:
    \begin{gather*}
        f_1=\set{(1,a),(2,x),(3,\text{Pepe})}\\
        f_2=\set{(1,a),(2,y),(3,\text{Pepe})}\\
        f_3=\set{(1,a),(2,z),(3,\text{Pepe})}\\
        f_4=\set{(1,b),(2,x),(3,\text{Pepe})}\\
        f_5=\set{(1,b),(2,y),(3,\text{Pepe})}\\
        f_6=\set{(1,b),(2,z),(3,\text{Pepe})}
    \end{gather*}

\task Sea la familia de conjuntos $\mathcal{F} = \set{\set{1,2}, \set{a,b,c}, \set{\text{Pepe}}}$.
    Si la función de indexación es la identidad $I: \mathcal{F} \to \mathcal{F}$ donde $I(A)=A$,
    tenemos que nuestro dominio es la propia familia $\mathcal{F}$ y el codominio es la $\bigcup_{A \in \mathcal{F}}A$.
    
    Así, los elementos del producto $\prod_{A \in \mathcal{F}} A$ son funciones $g$ que deben cumplir lo siguiente:
    \begin{itemize}
        \item $g(\set{1,2})$ debe ser $1$ o $2$.
        \item $g(\set{a,b,c})$ debe ser $a$, $b$ o $c$.
        \item $g(\set{\text{Pepe}})$ debe ser $\text{Pepe}$.
    \end{itemize}
    
    El producto cartesiano es, por tanto, el siguiente conjunto de $2 \times 3 \times 1 = 6$ funciones:
    \begin{gather*}
        g_1 = \set{(\set{1,2}, 1), (\set{a,b,c}, a), (\set{\text{Pepe}}, \text{Pepe})}\\
        g_2 = \set{(\set{1,2}, 1), (\set{a,b,c}, b), (\set{\text{Pepe}}, \text{Pepe})}\\
        g_3 = \set{(\set{1,2}, 1), (\set{a,b,c}, c), (\set{\text{Pepe}}, \text{Pepe})}\\
        g_4 = \set{(\set{1,2}, 2), (\set{a,b,c}, a), (\set{\text{Pepe}}, \text{Pepe})}\\
        g_5 = \set{(\set{1,2}, 2), (\set{a,b,c}, b), (\set{\text{Pepe}}, \text{Pepe})}\\
        g_6 = \set{(\set{1,2}, 2), (\set{a,b,c}, c), (\set{\text{Pepe}}, \text{Pepe})}
    \end{gather*}

\end{tasks}
\end{document}