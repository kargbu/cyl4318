\documentclass[fc]{tarea}

\usepackage[style=mexican]{csquotes}

\newcommand{\menos}{\backslash}
\newcommand{\al}{\symscr{A}}
\newcommand{\lan}{\symscr{L}}
\newcommand{\sis}{\symup{SF}}
\newcommand{\expresiones}{\symup{Exp}}
\NewDocumentCommand{\expr}{O{\al}}{\expresiones(#1)}

%conjuntos
\providecommand\st{}
\newcommand\SetSymbol[1][]{%
\nonscript\:#1\vert
\allowbreak
\nonscript\:
\mathopen{}}
\DeclarePairedDelimiterX\set[1]{\{}{\}}{%
\renewcommand\st{\SetSymbol[\delimsize]}#1}
\DeclarePairedDelimiterX\bset[1]{\lbrack}{\rbrack}{#1}

\examen{Tarea 2 (parte I)}
\prof{Karina G. Buendía y José Dosal}
\materia{Conjuntos y lógica}
\alum{}

%\xsimsetup{solution/print = true}

\begin{document}
\HojaExamen{}{e}
\begin{exercise}
 Sea $f: \mathcal{X} \to \mathcal{Y}$ una función. Demuestra que $F: \mathcal{P}(\mathcal{X}) \to \mathcal{P}(\mathcal{Y})$
 y $G: \mathcal{P}(\mathcal{Y}) \to \mathcal{P}(\mathcal{X})$ definidas como: $\begin{aligned}
    F(A) = f(A)\; \text{y}\; G(A) = f^{-1}(A),
 \end{aligned}$
son funciones.
\end{exercise}

\begin{solution}
Sean $(A, f(A))$ y $(B, f(B))$ en $F$. Por demostrar que $f(A)=f(B)$. Supongamos que $f(A) \neq f(B)$, entonces
$f(B) \menos f(A) \neq \emptyset$, es decir, existe $y \in f(B) \menos f(A)$. Por lo anterior, existe $x \in A$ tal que
$f(x) = y$, es decir $y \in f(A)$. Por lo tanto $f(A) = f(B)$
\end{solution}

\begin{exercise}
Sean $f: A \to C$ y $g:A \to B$ funciones. Demostrar que existe una función $h: B \to C$ tal que $f=h \circ g$
si y solo si para cada $x,\; y \in A$ $g(x) = g(y)$ implica $f(x) = f(y)$.
\end{exercise}

\begin{solution}
\textbf{($\Rightarrow$)} Supongamos que existe una función $h: B \to C$ tal que $f = h \circ g$.
Por demostrar que para cualesquiera $x, y \in A$, si $g(x) = g(y)$, entonces $f(x) = f(y)$.
Sean $x, y \in A$ tales que $g(x) = g(y)$,
$$f(x) = (h \circ g)(x) = h(g(x)) = h(g(y)) = (h \circ g)(y) = f(y)$$
Dado que $g(x) = g(y)$ y $h$ es función,
$$h(g(x)) = h(g(y))$$
Por lo tanto $f(x) = f(y)$.
\\
\textbf{($\Leftarrow$)} Supongamos que para todo $x, y \in A$ tenemos que $g(x) = g(y) \implies f(x) = f(y)$ es cierta.
Construyamos una función $h: B \to C$ tal que $f = h \circ g$. Sea $h$ definida para cualquier $b \in B$:
\begin{enumerate}
    \item Si $b \in \text{Im}(g)$, entonces existe al menos un $a \in A$ tal que $g(a) = b$. Para el cual definimos $h(b) = f(a)$.
      Si existiera otro elemento $a' \in A$ tal que $g(a') = b$, entonces tendríamos $g(a) = g(a')$. Por hipótesis, esto implica
      que $f(a) = f(a')$. Por lo tanto, el valor $h(b)$ es único.
    
    \item Si $b \notin \text{Im}(g)$, entonces no existe ningún $a \in A$ tal que $g(a)=b$. Así, $h(b) = c_0$. 
    Escogemos un elemento fijo $c_0 \in C$.
\end{enumerate}

\end{solution}

\begin{exercise}
   Sean $A \neq \emptyset$ y $B \neq \emptyset$ conjuntos. Para cualquier conjunto $C$ y cualesquiera funciones
   $f_1: C \to A$ y $f_2 : C \to B$ existe una única función $f: C \to A \times B$ tal que $f_1 = p_1 \circ f$
   y $f_2 = p_2 \circ f$. (Las funciones $f_1$ y $f_2$ se denominan funciones coordenadas)
\end{exercise}

\begin{solution}
   $\forall c \in C$ se tiene que $f_1(c) \in A$ y $f_2(c) \in B$. Definimos a $f(c) = (f_1(c), f_2(c))$.
   Es claro que es función ya que la pareja $(f_1(c), f_2(c))$ es única. Por demostrar que f cumple las 
   dos propiedades.
   \begin{enumerate}
      \item $\forall c \in C$ se tiene que $$ (p_1 \circ f)(c) = p_1(f(c)) = p_1(f_1(c), f_2(c)) = f_1(c)$$
            Por lo tanto, $p_1 \circ f = f_1$.
      \item $\forall c \in C$ se tiene que $$ (p_2 \circ f)(c) = p_2(f(c)) = p_2(f_1(c), f_2(c)) = f_2(c)$$
            Por lo tanto, $p_2 \circ f = f_2$. 
   \end{enumerate}
\end{solution}

\begin{exercise}
   Demuestra que si $I\neq \emptyset$ y algún $A_\alpha = \emptyset$ si y solo si $\prod_{\alpha \in I} A_{\alpha} = \emptyset$.
\end{exercise}

\begin{exercise}
   Sea $I \neq \emptyset$ un conjunto de índices. Considera dos familias indizadas $\set{A_{\alpha}}_{\alpha \in I}$ 
   y $\set{B_{\alpha}}_{\alpha \in I}$ . Demuestra lo siguiente:
   \begin{enumerate}
      \item Si $A_{\alpha} \subseteq B_{\alpha}$ para cada $\alpha \in I$, entonces
         $$\prod_{\alpha \in I} A_{\alpha} \subseteq \prod_{\alpha \in I} B_{\alpha}$$.
      \item El recíproco de (a) se cumple si $\prod_{\alpha \in I} A_{\alpha} \neq \emptyset$
   \end{enumerate}
\end{exercise}

\end{document}