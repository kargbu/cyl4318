\documentclass[fc]{tarea}

\usepackage[style=mexican]{csquotes}

\newcommand{\menos}{\backslash}
\newcommand{\al}{\symscr{A}}
\newcommand{\lan}{\symscr{L}}
\newcommand{\sis}{\symup{SF}}
\newcommand{\expresiones}{\symup{Exp}}
\NewDocumentCommand{\expr}{O{\al}}{\expresiones(#1)}

%conjuntos
\providecommand\st{}
\newcommand\SetSymbol[1][]{%
\nonscript\:#1\vert
\allowbreak
\nonscript\:
\mathopen{}}
\DeclarePairedDelimiterX\set[1]{\{}{\}}{%
\renewcommand\st{\SetSymbol[\delimsize]}#1}
\DeclarePairedDelimiterX\bset[1]{\lbrack}{\rbrack}{#1}

\examen{Tarea 2}
\prof{Karina G. Buendía y José Dosal}
\materia{Conjuntos y lógica}
\alum{}

\xsimsetup{solution/print = true}

\begin{document}
\HojaExamen{}{e}
\begin{exercise}
 Sea $f: \mathcal{X} \to \mathcal{Y}$ una función. Demuestra que $F: \mathcal{P}(\mathcal{X}) \to \mathcal{P}(\mathcal{Y})$
 y $G: \mathcal{P}(\mathcal{Y}) \to \mathcal{P}(\mathcal{X})$ definidas como: $\begin{aligned}
    F(A) = f(A),\; G(A) = f^{-1}(A),
 \end{aligned}$
son funciones.
\end{exercise}

\begin{solution}
Sean $(A, f(A))$ y $(B, f(B))$ en $F$. Por demostrar que $f(A)=f(B)$. Supongamos que $f(A) \neq f(B)$, entonces
$f(B) \menos f(A) \neq \emptyset$, es decir, existe $y \in f(B) \menos f(A)$. Por lo anterior, existe $x \in A$ tal que
$f(x) = y$, es decir $y \in f(A)$. Por lo tanto $f(A) = f(B)$
\end{solution}

\begin{exercise}
Sea $f: A \to B$
\end{exercise}

\begin{solution}

\end{solution}

\end{document}