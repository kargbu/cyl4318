\documentclass[fc]{tarea}
\usepackage{amsthm}

\newtheorem{lema}{Lema}

\usetheme{Madrid}

\title{Teoría de Conjuntos}
\author{Karina García Buendía}
\date{\today}

\begin{document}

\begin{frame}
\titlepage
\end{frame}

\begin{frame}{Axiomas}
    \begin{block}{Axioma 1: Existencia}
        \textit {Existe un conjunto que no tiene elementos.}
    \end{block}

    \begin{block}{Axioma 2: Extensión}
         \textit {Si todo elemento de $X$ es elemento de $Y$ y todo elemento de $Y$ es elemento de $X$, entonces
         $X =Y$.}
    \end{block}

        \begin{block}{Axioma 3: Esquema de comprensión}
            \textit {Sea una propiedad $P(x)$. Para cualquier conjunto $A$ hay un conjunto $B$ tal que $x \in B \Leftrightarrow$
            $x \in A$ y $P(x)$.} 
        \end{block}
\end{frame}

\begin{frame}
    Ejemplo 1. Si $P$ y $Q$ son conjuntos, entonces hay un conjunto $R$ tal que $x \in R$
    si y solo si $x \in P$ y $x \in Q$.
\pause
    \begin{proof}
        Consideremos la propiedad de que $x \in Q$, es decir $P(x,Q)$. Por el axioma esquema de comprensión se tiene
        que para todo $Q$ y cualquier $P$ hay un conjunto $R$ tal que $x \in R$ si y solo si $x \in P$ y $x \in Q$.
    \end{proof}    
\pause
    Ejemplo 2. El conjunto de todos los conjuntos no existen.
    \begin{proof}
        Supongamos lo contrario.
    \end{proof}
\end{frame}

\begin{frame}
    \begin{lema}
        Sea $P(x)$ una propiedad de $x$. Para todo conjunto $A$ hay un único conjunto $B$ tal que $x \in B$
        si y solo si $x \in A$ y $P(x)$
    \end{lema}
\end{frame}

\begin{frame}
        \begin{block}{Axioma 4: del Par}
            \textit {Para cualesquiera $a$ y $b$ existe un conjunto $C$ tal que $x \in C$ si y solo si $x = a$ o $x = b$.}
        \end{block}

\pause

\vspace{1em}
     ¿Todo conjunto es un elemento de algún otro conjunto?, ¿dos conjuntos cuales quiera son simultáneamente elementos
     de algún mismo conjunto?
\end{frame}

\begin{frame}{Axiomas}
    \begin{block}{Axioma 5: de Unión}
        \textit {Para cualquier $S$, existe un conjunto $U$ tal que si $x \in U$ si y solo si $x \in X$ para algún $X \in S$.}
    \end{block}

\vspace{1em}

    \begin{itemize}
        \item Ejemplo 3. Sea $S=\{\emptyset, \{\emptyset\}\}$
        \pause
        ¿Quién es $\bigcup S$?
        \pause
        \vspace{1em}

        Entonces $x \in \bigcup S$ si y solo si $x \in A$ para algún conjunto $A$ en $S$. Es decir, si y solo si $x \in \emptyset$
        o $x \in \{\emptyset\}$
        \item $\bigcup \emptyset = \emptyset$
        \item Sean $A$ y $B$ conjuntos, existe $\bigcup \{A, B \}$ tal que $x \in \bigcup \{A, B \}$ si y solo si
        $x \in A$ o $x \in B$. 
    \end{itemize}

\end{frame}

\end{document}
